
\begin{thanks}
\zihao{5}

这篇论文是在董海玲副教授的悉心关怀和指导下完成的. 从论文的选题、收集材料到最后论文的撰写及定稿, 无不凝聚着董老师的心血. 同时
很庆幸在这三年的研究生生涯中遇到董老师, 无论是学习和生活, 她都给予我极
大的关心和帮助, 她严谨治学的态度、宽以待人的品格、乐观进取的心态、平易近
人的长者风范无时无刻不在影响着我. 董老师渊博的学识、坚实的专业知识和敏锐
的洞察力使我受益非浅, 将是我一生学习的榜样. 至此论文完成之际, 向董老师致
以最诚挚的谢意.

衷心感谢审阅鄙人论文的专家学者们, 谢谢您在百忙之中抽空关注晚辈的学习成果, 同时也感谢阅读过我论文的同行人或关注者, 谢谢您的关注和建议!

衷心感谢讨论班的所有成员, 他们是丰建文教授、赵毅老师、董海玲老师、王劲毅大师兄、余芬芬师姐、吴维扬师姐、叶丹凤师姐、杨攀同学、李娜同学、付方方同学.
一起生活学习的日子使我终生难忘!

衷心感谢任课老师魏正红老师、张君老师、李松臣老师、温松桥老师、蒋春福老师、姚念老师、张玉老师、曹丽华老师、王启华老师、丰建文老师、赵毅老师等, 您们无私的传授知识给我, 恩师难忘, 我的每一步成长都离不开师长的教导和呵护。

衷心感谢林俊生、莫广焰、楚天玥、孙明以及数学专业的同学们, 特别是球友们和饭友们, 是你们陪我一起在这座美丽的校园里度过快乐的三年时光, 我们在相互鼓励中不断成长、进步, 留下了许多难忘的回忆. 如今, 大家都各奔东西, 衷心祝愿大家在今后的人生道路上事事顺心, 勇于不断攀登生活和事业得高峰, 幸福美满. 我们的同窗之谊永存.



%感谢CRAN, CTAN, Linux, Matlab的开发者及其成员, 你们的努力为我论文的完成提供了必要的软件资源. 谢谢你们!

最后, 我要感谢我的父母一直以来给予的支持和信任, 他们用最无私的奉献和真挚的爱支持我顺利完成学业, 他们为我的成长付出了许多许多, 焉得谖草, 言树之背, 养育之恩, 无以回报, 惟愿他们健康长寿.

由于作者水平有限, 论文中难免存在错误或不妥之处, 恳请给位专家教授指正.

\vskip 18pt
~~~~~~~~~~~~~~~~~~~~~~~~~~~~~~~~~~~~~~~~~~~~~~~~~~~~~~~~~~~~~~~~~~~~~~~~~~~~~~~~~~~~~~~~~~~~~周家木
\vskip 10pt
~~~~~~~~~~~~~~~~~~~~~~~~~~~~~~~~~~~~~~~~~~~~~~~~~~~~~~~~~~~~~~~~~~~~~~~~~~~2017~年~3~月~6~日~于~深圳大学

\end{thanks}
