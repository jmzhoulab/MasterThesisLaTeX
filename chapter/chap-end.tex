%!Mode:: "Tex:UTF-8"
\chapter{总结与展望}\label{chap:end}

本学位论文主要运用事件激发控制策略分别讨论了三类处于不同随机环境下带有马氏切换耦合复杂网络同步问题.

首先讨论了非线性耦合拓扑与随机发生的耦合以及随机耦合强度并存的复杂网络均方指数同步问题. 适当选择马尔可夫链处于不同状态下的牵制节点集, 并分别在连续监控和离散监控两种情形下构造两个基于不同测量误差上界的事件激发规则, 根据这些激发规则和控制器相结合形成的事件激发牵制控制策略来实现网络同步. 在定理中给出了判定同步的判据, 并通过构造简洁的Lyapunov-Krasovskii函数, 利用稳定性理论和比较原理给出指数同步判据是充分有效的. 最后给出一个节点相对较大规模的复杂网络数值例子证实理论结果的有效性.

其次 考察了时滞与部分转移率共存的复杂网络同步问题. 实际应用由于网络带宽或者有限速度的影响, 节点动力学行为存在时滞在所难免, 因此研究时滞的网络系统是很有必要的. 为了实现网络在无具体同步目标的情况下实现网络的同步, 在理论证明中设定合适的参考节点作为同步目标. 通过事件激发策略和随机控制方法, 得出了网络同步的充分条件. 并且在同步的基础上进一步证明该事件激发策略下的控制方法不会引起Zeno现象.

最后 研究了带有噪声干扰的部分转移率未知的马尔可夫切换拓扑复杂网络同步问题. 这里建立了数学模型模拟实际环境中存在的噪声影响, 同时考虑到马尔可夫链的转移率可能部分未知. 这里的噪声主要是考虑信号传输过程中受到的干扰, 即节点与节点间通讯受到干扰. 与前面不同的是, 这里考虑连续监控下的集中式和分散式的事件激发策略. 根据节点的测量误差与同步误差构造相应的激发函数, 激发函数是否到达阈值$0$即为事件激发规则. 结合马氏调制的牵制控制方法, 分别在两种事件激发策略实现了噪声影响下部分转移率的复杂网络均方同步.根据Lyapunov稳定性定理与随机分析理论得到网络同步的充分条件.

回顾网络科学理论的发展历程, 尽管复杂网络的研究热潮刚兴起不久, 但有关复杂网络的科研成果不胜枚举. 随着社会发展、科技进步, 特别是在大数据、互联网+时代, 网络结构的关系更加复杂, 节点维度急剧增加, 复杂网络在各个领域将面临着新的挑战. 这同时也是推动复杂网络向前发展的绝世良机. 目前, 本人觉得一下几个方面值得深入和广泛的研究:
\begin{itemize}\setlength{\itemsep}{0cm}
  \item 如何从单一网络控制方法应用到多重网络(即网络的网络)中;
  \item 当网络的切换拓扑是带跳过程时又该如何构造控制方法使得网络同步;
  \item 如何从现有的网络结构挖掘出缺失的信息;
  \item 如何通过现有的节点状态信息预测节点在未来某时刻的行为特征.
\end{itemize}



